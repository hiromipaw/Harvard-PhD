Las comunicaciones en línea generan una cantidad constante de datos que fluyen entre usuarios, servicios y aplicaciones. Esta información es el resultado de las interacciones entre diferentes partes y, una vez recolectada, se utiliza para una gran variedad de propósitos, desde perfiles de marketing hasta recomendaciones de productos, pasando por filtros de noticias y sugerencias de relaciones. La motivación detrás de este trabajo es entender cómo los datos son compartidos y utilizados por los servicios en nombre de los usuarios. Cuando un usuario crea una nueva cuenta en una determinada plataforma, ello crea un contenedor lógico que se utilizará para almacenar la actividad del propio usuario. El servicio tiene como objetivo perfilar al usuario. Por lo tanto, cada vez que se crean, se comparten o se accede a los datos, se recopila y analiza información sobre el comportamiento y los intereses del usuario. Los usuarios producen estos datos pero desconocen cómo serán manejados por el servicio, o con quién se compartirán. O lo que es más importante, una vez agregados, estos datos podrían revelar, con el tiempo, más información de la que los mismos usuarios habían previsto inicialmente. La información revelada por un perfil podría utilizarse para obtener acceso a otra cuenta o durante ataques de ingeniería social. El objetivo principal de esta tesis es modelar y analizar cómo fluyen los datos de los usuarios entre diferentes aplicaciones y cómo esto representa una amenaza importante para la privacidad. Con el propósito de definir las violaciones de privacidad, se utilizan patrones que permiten clasificar las amenazas e identificar los problemas en los que los datos de los usuarios son mal gestionados. Los datos de los usuarios se modelan como eventos categorizados y se agregan como histogramas de frecuencias relativas de actividad en línea en categorías predefinidas de intereses. Además, se introduce un paradigma basado en hipermedia para modelar las huellas en línea. Esto enfatiza la interacción entre los diferentes eventos generados por el usuario y sus efectos sobre el riesgo medido de privacidad del usuario. Finalmente, se discuten las lecciones aprendidas de la aplicación del paradigma a diferentes escenarios.