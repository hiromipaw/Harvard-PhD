On-line communications generate a consistent amount of data flowing between users, services and applications. This information results from the interactions between different parties and once collected is used for a variety of purposes, from marketing profiling, to product recommendations, from news filtering to relationships suggestions. 

Understanding how data is shared and used by services on behalf of users is the motivation behind this work.  When a user creates a new account on a certain platform, this creates a logical container that will use to store the user's activity. The service aims at profiling the user, therefore every time some data is created, shared or accessed, information about the user behaviour and interests is collected and analysed. Users produce this data but are unaware of how this will be handled by the service, nor who it will be shared with. More importantly, once aggregated, these data could reveal more over time that the same users initially intended. Information revealed by one profile could be used either to obtain access to another account or during social engineering attacks.

The main focus of this dissertation is modelling and analysing how user data flows between different application and how this represent an important threat for privacy. A framework defining privacy violation is used to classify threats and identify issues where user data are effectively mishandled. Users data is modelled as categorised events, and aggregated as histograms of relative frequencies of on-line activity along predefined categories of interests. Furthermore, a paradigm based on hypermedia to model on-line footprints is introduced. This emphasises the interactions between different user-generated events and their effects on the user’s measured privacy risk. Finally, lesson learnt from applying the paradigm to different scenarios are discussed.