Online communications generate a consistent amount of data
flowing among users, services and applications. This
information results from the interactions between different
parties, and once collected, it is used for a variety of
purposes, from marketing profiling to product recommendations,
from news filtering to relationship suggestions.
Understanding how data is shared and used by services on behalf
of users is the motivation behind this work. When a user creates
a new account on a certain platform, this creates a logical
container that will be used to store the user's activity. The
service aims to profile the user. Therefore, every time some
data is created, shared or accessed, information about the
user’s behaviour and interests is collected and analysed. Users
produce this data but are unaware of how it will be handled by
the service, and of whom it will be shared with. More
importantly, once aggregated, this data could reveal more over
time that the same users initially intended. Information
revealed by one profile could be used to obtain access to
another account, or during social engineering attacks.
The main focus of this dissertation is modelling and analysing
how user data flows among different applications and how this
represents an important threat for privacy. A framework defining
privacy violation is used to classify threats and identify
issues where user data is effectively mishandled. User data is
modelled as categorised events, and aggregated as histograms of
relative frequencies of online activity along predefined
categories of interests. Furthermore, a paradigm based on
hypermedia to model online footprints is introduced. This
emphasises the interactions between different user-generated
events and their effects on the user’s measured privacy risk.
Finally, the lessons learnt from applying the paradigm to
different scenarios are discussed.