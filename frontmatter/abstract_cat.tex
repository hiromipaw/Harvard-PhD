Les comunicacions en línia generen una quantitat constant de dades que flueixen entre usuaris, serveis i aplicacions. Aquesta informació és el resultat de les interaccions entre diferents parts i, un cop recol·lectada, s'utilitza per a una gran varietat de propòsits, des de perfils de màrqueting fins a recomanacions de productes, passant per filtres de notícies i suggeriments de relacions. La motivació darrere d'aquest treball és entendre com les dades són compartides i utilitzades pels serveis en nom dels usuaris. Quan un usuari crea un nou compte en una determinada plataforma, això crea un contenidor lògic que s'utilitzarà per emmagatzemar l'activitat del propi usuari. El servei té com a objectiu perfilar a l'usuari. Per tant, cada vegada que es creen, es comparteixen o s'accedeix a les dades, es recopila i analitza informació sobre el comportament i els interessos de l'usuari. Els usuaris produeixen aquestes dades però desconeixen com seran gestionades pel servei, o amb qui es compartiran. O el que és més important, un cop agregades, aquestes dades podrien revelar, amb el temps, més informació de la que els mateixos usuaris havien previst inicialment. La informació revelada per un perfil podria utilitzar-se per accedir a un altre compte o durant atacs d'enginyeria social. L'objectiu principal d'aquesta tesi és modelar i analitzar com flueixen les dades dels usuaris entre diferents aplicacions i com això representa una amenaça important per a la privacitat. Amb el propòsit de definir les violacions de privacitat, s'utilitzen patrons que permeten classificar les amenaces i identificar els problemes en què les dades dels usuaris són mal gestionades. Les dades dels usuaris es modelen com esdeveniments categoritzats i s'agreguen com histogrames de freqüències relatives d'activitat en línia en categories predefinides d'interessos. A més, s'introdueix un paradigma basat en hipermèdia per modelar les petjades en línia. Això emfatitza la interacció entre els diferents esdeveniments generats per l'usuari i els seus efectes sobre el risc mesurat de privacitat de l'usuari. Finalment, es discuteixen les lliçons apreses de l'aplicació del paradigma a diferents escenaris.