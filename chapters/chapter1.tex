\begin{savequote}[75mm] 
It is poor civic hygiene to install technologies that could someday facilitate a police state.
\qauthor{Bruce Schneier} 
\end{savequote}

\chapter{Introduction}

\newthought{On-line communications} are increasingly opening new possibilities for people to access and create content and interact with one another on the web. On the one hand web application facilitate access to information and foster relationships creation. On the other hand, as networking systems are constantly evolving, and on-line interactions are becoming more frequents and complex, it is becoming impossible to retain control over what is perceived as our on-line footprint. More specifically, user can share data with different services, that can subsequently share this information to third parties, sometimes without asking for permission to do so. Third parties are entitled to retain data over time, even if they have no direct connection with the user of the original service. More over it has become a general practice to share content on different platforms and applications simultaneously. Such behaviour creates multiple possibilities for users to be potentially targets of various attacks and different profiling activities.

Up to now, in an on-line context, the right to privacy has commonly been interpreted as a right to \emph{information self-determination}. Acts typically claimed to breach on-line privacy concern the collection of personal information without consent, the selling of personal information and the further processing of that information. This definition of privacy breach can be considered valid until the user has direct control of the data they have created. This is not always the case. In 2011, the amount of digital information created and replicated globally exceeded 1.8 zettabytes (1.8 trillion gigabytes). 75\% of this information is created by individuals through new media fora such as blogs and via social networks. By the end of 2011, Facebook had 845 million monthly active users, sharing over 30 billion pieces of content~\cite{library-briefing}. Three quarters of the 1.8 trillion gigabytes of digital information on-line has been created by individual users. On top of that, an increasing amount of additional data about those users is collected by public and private companies, for the most disparate range of uses.

This dissertation is motivated by understanding how data, created by users, flows between applications and services. A very powerful example in this field is the use of federated log in mechanisms. To register to a new social applications, users grant them a certain level of access to their identity data, through, for example, their Facebook, Twitter or Google accounts. This data include details about their identity, their whereabouts and in some situations even the company they work for. Third parties, like Facebook or Google, offer log in technologies, allowing the application to identify the user and receive precise information about them. Once the user grant access to their data, the application stores it and assumes control over how it is further shared. The user will never be notified again on who is accessing their data, nor if these are transferred to third parties. 

The focus of this work is exploring the intersection between accurately modelling users' interactions and expressing private information in a way that it is possible to compute a numerical estimation of its impact for the user privacy. This first chapter presents a literature review of the problems considered throughout this work.

The second chapter introduces users' profile modelling and Privacy Enhancing Techniques in the field of social tagging systems. This chapter is particularly concerned with understanding how recommendations algorithms react to profile perturbation and how the utility of the algorithm is affected.

The third chapter is centred on how proximity-based social applications and the idea of serendipitous discovery of interests, places and social connections can be exploited by potential attackers. It is analysed how this services allow users to interact with people that are currently close to them, by revealing some information about their preferences and whereabouts. This information is acquired through passive geo-localisation and used to build a sense of serendipity. Unfortunately, while this class of applications opens different interactions possibilities for people in urban settings, obtaining access to certain identity information could lead a possible privacy attacker to identify and follow a user in their movements in a specific period of time. The same information shared through the platform could also help an attacker to link the victim’s on-line profiles to physical identities. This chapter is also concerned with the possibilities presented by mobile devices to act as listening sensors and how these could eventually lead to newer privacy attacks.

The fourth chapter is focused on web tracking and how advertising networks are able \emph{to follow} users while they surf the web. In this chapter it is highlighted the shift in the evolution of the Internet, from a stage when when web sites were just hypertext documents, with no personalisation of the user experience offered, to the web of today, a world wide distributed system following specific architectural paradigms. Nowadays, an enormous quantity of user-generated data is shared and consumed by a network of applications and services, reasoning upon users expressed preferences and their social and physical connections. Advertising networks follow users’ browsing habits while they surf the web, continuously collecting their traces and surfing patterns. We analyse how users tracking happens on the web by measuring their on-line footprint and estimating how quickly advertising networks are able to profile users by their browsing habits.

In the fifth chapter it is explored how user profile change every time a user publishes a new post or creates a link with another entity, either another user, or some on-line resource. When new information is added to the user profile, new private data is exposed. This doesn't only reveal information about single users' preferences, increasing their privacy risk, but can expose more about their network that single actors intended. This mechanism is self-evident on \emph{social networks} where users receive suggestions based on their friends activity. An information theoretic approach to measure the differential update of the anonymity risk for time variant users profiles is proposed. This expresses how privacy is affected when new content is posted and how much third party services \emph{get to know} about the users when a new activity is shared. We use real Facebook data to show how our model can be applied on a real world scenario.

In the sixth chapter it is presented a hypermedia model of the user on-line footprint. This model consider the architectural paradigms of the web and applies them to modelling of private information and especially on how this can be exchanged with a certain level of user control. We analyse the current models to grant access to private data and how this could be modified in order to achieve a better user supervision over their footprints. Furthermore we analyse how this data could be applied with user willing to grant access to third-party apps to their Facebook profile in exchange for some service.

Finally, the seventh chapter will be reserved for conclusions and final discussions regarding results of this dissertation.

In summary, this dissertation makes the following contributions to research within the field of Information Privacy:

\begin{enumerate}
    \item An analysis of how PETs affect recommendation systems for social tagging platforms.
    \item An analysis of privacy risks for proximity based social applications.
    \item An analysis of how users are tracked while surfing the web.
    \item An information theoretic approach to measure the differential update of the anonymity risk for time variant users profiles.
    \item A hypermedia model of the user on-line footprint and a privacy management solution.
\end{enumerate}